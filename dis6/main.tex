\documentclass[letterpaper,12pt]{article}

% Recommended, but optional, packages for figures and better typesetting:
\usepackage{microtype}
\usepackage{graphicx}
% \usepackage{subfigure}
\usepackage{booktabs} % for professional tables

% hyperref makes hyperlinks in the resulting PDF.
% If your build breaks (sometimes temporarily if a hyperlink spans a page)
% please comment out the following usepackage line and replace
% \usepackage{icml2023} with \usepackage[nohyperref]{icml2023} above.
% \usepackage{hyperref}
\usepackage{natbib}
\usepackage[dvipsnames,svgnames,table]{xcolor}
\usepackage{multicol}

% Attempt to make hyperref and algorithmic work together better:
\newcommand{\theHalgorithm}{\arabic{algorithm}}

\usepackage{arxiv}

% For theorems and such
\usepackage{amsmath}
\usepackage{amssymb}
\usepackage{mathtools}
\usepackage{amsthm}

\input{math_commands.tex}

% if you use cleveref..

%%%%%%%%%%%%%%%%%%%%%%%%%%%%%%%%
% THEOREMS
%%%%%%%%%%%%%%%%%%%%%%%%%%%%%%%%
\theoremstyle{plain}
\newtheorem{theorem}{Theorem}[section]
\newtheorem{proposition}[theorem]{Proposition}
\newtheorem{lemma}[theorem]{Lemma}
\newtheorem{corollary}[theorem]{Corollary}
\theoremstyle{definition}
\newtheorem{definition}[theorem]{Definition}
\newtheorem{assumption}[theorem]{Assumption}
\theoremstyle{remark}
\newtheorem{remark}[theorem]{Remark}

% Todonotes is useful during development; simply uncomment the next line
%    and comment out the line below the next line to turn off comments
\usepackage[disable,textsize=tiny]{todonotes}
% \usepackage[textsize=tiny]{todonotes}

\usepackage{amsfonts}
\usepackage[utf8]{inputenc}
\usepackage{multirow}
\usepackage{makecell}
\usepackage[pagebackref]{hyperref}
\hypersetup{
    colorlinks=true,
    allcolors=DarkBlue
%   allbordercolors=red
}
\usepackage{courier}
\usepackage{xspace}
\usepackage{bm}
\usepackage{wrapfig}
\usepackage{oubraces}
\usepackage{amsthm}
\usepackage[linesnumbered,lined,boxed,commentsnumbered]{algorithm2e}
\usepackage{algorithmic}
\usepackage{caption}
\usepackage{subcaption}
\usepackage{stfloats}
\usepackage{url}
\usepackage{enumitem}
\usepackage[capitalize,noabbrev]{cleveref}

\RestyleAlgo{ruled}
\definecolor{Gray}{gray}{0.7}
\newcolumntype{g}{>{\columncolor{Gray}}r}

\def\Ex{\mathop{\mathbb{E}}}
\def\P{\mathop{\mathrm{P}}}
\newcommand{\norm}[1]{\left\lVert#1\right\rVert}
\newcommand{\abs}[1]{\left|#1\right|}
\newcommand{\sgn}[1]{\text{sign}\left(#1\right)}
\newcommand{\inner}[1]{\left\langle#1\right\rangle}
\def\minop{\mathop{\rm min}\limits}
\def\maxop{\mathop{\rm max}\limits}
\newcommand{\ber}[1]{\mathrm{Bern}\left(#1\right)}
\def\unif{\mathcal{U}}
\def\w{\vw}
\def\X{\mX}
\def\y{\vy}
\def\x{\vx}
\def\eqref#1{Eqn.~(\ref{#1})}
\def\figref#1{Fig.~\ref{#1}}

% \crefname{algocf}{alg.}{algs.}
\Crefname{algocf}{Algorithm}{Algorithms}

\newcommand{\chawin}[1]{\textcolor{green}{Chawin: #1}}
% \newcommand{\chawin}[1]{}
\newcommand{\note}[1]{\textcolor{blue}{Note: #1}}
% \newcommand{\todo}[1]{\textcolor{red}{TODO: #1}}
\newcommand{\question}[1]{\textcolor{orange}{\textit{#1}}}


\renewcommand{\algorithmiccomment}[1]{{\color{gray}\small \# \emph{#1}}}

% The \icmltitle you define below is probably too long as a header.
% Therefore, a short form for the running title is supplied here:
% \icmltitlerunning{Realistic Decision-Based Attacks on Machine Learning Systems\hfill\thepage}

\hypersetup{
pdftitle={CS189/289A Spring 2023},
pdfsubject={cs.CR,cs.CV,cs.LG},
pdfauthor={Chawin Sitawarin},
pdfkeywords={},
}


\title{CS189/289A Spring 2023\\Least-Norm Solution of Least Square Problem}

\author{
    Chawin Sitawarin \thanks{Corresponding email: \texttt{chawins@berkeley.edu}}
}

\date{
    University of California, Berkeley
}

\begin{document}

\maketitle

Demo: \url{https://colab.research.google.com/drive/1AtUrdCJkDr2P8CSTYd_evvm6J14Da7rr}

\section{What is Least-Norm Solution?}

In this discussion section, we will go over the Problem 1 and 2 in the worksheet which introduces and proves some basic properties of the least-norm solution of the least square problem (its uniqueness, the pseudo-inverse, etc.).

\section{SGD Finds the Least-Norm Solution}

The demo will show that (stochastic) gradient descent is guaranteed to find the least-norm solution when the weight is initialized as zero.
The intuition is that all the gradient updates live in the row space of the data matrix $\mX$, and the least-norm solution is the only solution in the row space.

% \bibliography{reference.bib}
% \bibliographystyle{abbrvnat}

%%%%%%%%%%%%%%%%%%%%%%%%%%%%%%%%%%%%%%%%%%%%%%%%%%%%%%%%%%%%%%%%%%%%%%%%%%%%%%%
%%%%%%%%%%%%%%%%%%%%%%%%%%%%%%%%%%%%%%%%%%%%%%%%%%%%%%%%%%%%%%%%%%%%%%%%%%%%%%%
% APPENDIX
%%%%%%%%%%%%%%%%%%%%%%%%%%%%%%%%%%%%%%%%%%%%%%%%%%%%%%%%%%%%%%%%%%%%%%%%%%%%%%%
%%%%%%%%%%%%%%%%%%%%%%%%%%%%%%%%%%%%%%%%%%%%%%%%%%%%%%%%%%%%%%%%%%%%%%%%%%%%%%%
\newpage
\appendix
\onecolumn
%%%%%%%%%%%%%%%%%%%%%%%%%%%%%%%%%%%%%%%%%%%%%%%%%%%%%%%%%%%%%%%%%%%%%%%%%%%%%%%
%%%%%%%%%%%%%%%%%%%%%%%%%%%%%%%%%%%%%%%%%%%%%%%%%%%%%%%%%%%%%%%%%%%%%%%%%%%%%%%

\end{document}


% This document was modified from the file originally made available by
% Pat Langley and Andrea Danyluk for ICML-2K. This version was created
% by Iain Murray in 2018, and modified by Alexandre Bouchard in
% 2019 and 2021 and by Csaba Szepesvari, Gang Niu and Sivan Sabato in 2022.
% Modified again in 2023 by Sivan Sabato and Jonathan Scarlett.
% Previous contributors include Dan Roy, Lise Getoor and Tobias
% Scheffer, which was slightly modified from the 2010 version by
% Thorsten Joachims & Johannes Fuernkranz, slightly modified from the
% 2009 version by Kiri Wagstaff and Sam Roweis's 2008 version, which is
% slightly modified from Prasad Tadepalli's 2007 version which is a
% lightly changed version of the previous year's version by Andrew
% Moore, which was in turn edited from those of Kristian Kersting and
% Codrina Lauth. Alex Smola contributed to the algorithmic style files.
